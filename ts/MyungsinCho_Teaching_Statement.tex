\documentclass[11pt]{article}
\usepackage[letterpaper,margin=1in]{geometry}
\usepackage{hyperref}

\usepackage{newtxtext, kotex}

\usepackage{fancyhdr}
%\fancyhf{} % sets both header and footer to nothing
\renewcommand{\headrulewidth}{0pt}


\usepackage[textwidth=1in,textsize=small,colorinlistoftodos]{todonotes} % todo 노트

\pagestyle{fancy}
\lhead{Myungsin Cho\\ }
%\chead{\LARGE Teaching Statement}
%\rhead{\univ \\ Ph.D. Mathematics}

% section 폰트 바꿔주는거
\usepackage{titlesec}
\titleformat{\section}{\normalfont\large\center}{\thesection}{.5em}{}
%\titleformat{\subsection}[runin]{\normalfont\bfseries}{\thesubsection}{.5em}{}

\newcommand{\yourname}{Myungsin Cho}

\newcommand\quelle[1]{{%
      \unskip\nobreak\hfil\penalty50
      \hskip2em\hbox{}\nobreak\hfil #1%
      \parfillskip=0pt \finalhyphendemerits=0 \par}}
      
\title{Research Statement}
\author{Myungsin Cho}
\date{}
\begin{document}
\begin{center}\LARGE{Teaching Statement}\end{center}

My central goal as an educator is to help students think mathematically—to frame problems, reason with definitions, and connect methods to broader ideas.
I emphasize understanding through concrete examples before memorization: once a student understands the meaning of a definition, the associated techniques follow naturally.
I want students to see mathematics not just as a collection of rules, but as a coherent way of reasoning.

For instance, I frame calculus through the unifying theme of approximation. 
What may seem to students as separate topics can instead be seen as parts of a single theme, offering coherence and intuitive insight into the subject’s central properties. 
This approach reduces the pressure to memorize isolated formulas and highlights the larger narrative of the subject. 
As a result, students tend to engage more actively when they recognize how the pieces fit together.

\paragraph{Large-Lecture and Applied Contexts}\quad \\
Most recently, I am teaching {\it Analysis and Optimization} at Columbia University, a large-lecture course taken mainly by students in applied mathematics and other STEM fields. 
Unlike proof-based analysis courses, this class emphasizes applications and computation. 
I prepare lectures around a cycle of concepts $\to$ example $\to$ practice, using carefully chosen examples that draw from economics, engineering, and social sciences.
Presenting abstract ideas through concrete situations keeps the material accessible while still highlighting the reasoning behind the formulas.

Although the course syllabus is departmentally coordinated, I design my lectures independently rather than following the textbook line by line.
This flexibility helps me adjust explanations and choose applications that connect with students’ backgrounds.
I also write homework problems myself rather than assigning them directly from the textbook. 
By aligning problem sets closely with the themes of my lectures, students gain a stronger sense of continuity and are better able to apply what they have learned in class.
My own undergraduate training in engineering has been an asset here: I understand how STEM students approach technical problems, and I can present mathematical ideas in ways that feel natural to them. 
This experience has strengthened my ability to manage large and varied classrooms while keeping lectures clear and well structured.

Across these different settings, I have learned to adjust both the level of abstraction and the pace of instruction to match the audience while keeping the emphasis on active engagement.

\paragraph{Adapting to Different Learning Environments} \quad \\
My teaching spans institutions with very different student populations.
At Indiana University, I taught Calculus I as the primary instructor for a section of about 70 students.
With wide variation in preparation, I used short quizzes, carefully designed homework assignments, and structured review sessions to build confidence and strengthen core understanding.
Students responded especially positively to the homework, noting in evaluations that it helped them connect lecture material with practice.
The withdrawal rate in my section was about \emph{10\% lower than the departmental average}, one of the lowest among twelve instructors that semester. Teaching evaluations also showed that my scores were consistently about \emph{10\% above the departmental average across most categories}, with students frequently mentioning clear explanations and an approachable class atmosphere.

These strategies were partly shaped by my earlier experience at Seoul National University, where I served as a TA for a calculus course with students who generally had stronger preparation. 
Although the content was similar, I adjusted my approach: instead of focusing on review and confidence-building, I emphasized justification and comparison of methods. 
I encouraged students not only to solve problems but also to explain why their solutions were valid and how they related to alternative approaches. 
Short, structured discussions gave them practice in reasoning and proof.
During this time, I received a \emph{Lecture and Research Scholarship}, awarded in recognition of both strong teaching performance and research potential.
Across these experiences, I learned to adjust my teaching to local contexts while keeping active participation central.


\paragraph{Mentoring and Individualized Support}\quad \\
I place high value on mentoring and individualized guidance.
Through the \emph{Directed Reading Program} at Indiana, I worked with undergraduates on advanced topics ranging from algebraic topology to category theory.
I set weekly milestones, asked students to produce short write-ups, and encouraged them to present their findings in accessible formats.
These projects often gave students the confidence to enroll in graduate-level courses the following semester.
The process of setting goals and receiving regular feedback helped them take on more advanced work and develop independent learning skills.

I also view office hours and one-on-one conversations as essential for supporting students who might otherwise feel left behind. 
I make a point of encouraging them individually, whether by checking in before class, clarifying that all questions are welcome, or reaching out when a student is struggling with coursework.
My classrooms often include students from a wide range of majors and backgrounds, and I aim to make mathematics approachable to all of them through individual interaction and steady encouragement.


\paragraph{Assessment and Feedback}\quad \\
I view assessments as part of the learning process rather than as one-time events. To support growth, I let students revisit and correct mistakes on exams, awarding partial credit for thoughtful revisions. This shifts the focus from performance to improvement and eases the pressure of high-stakes testing. I also design extra-credit assignments when appropriate: enrichment projects for students who want to explore beyond the syllabus, or additional problems for those who need more practice. I have found that these adjustments lead to more consistent engagement and clearer improvement over the semester, especially among students who initially struggled.



\paragraph{Conclusion} \quad \\
For many students, my class may be their last formal encounter with mathematics. I want them to leave with confidence in their abilities, a clear grasp of key ideas, and an appreciation of how mathematics connects to the wider world. For mathematics majors, I aim to help them build habits of proof and reasoning that prepare them for advanced study and research. Whether in a large applied lecture or a small reading group, I see teaching as guiding students toward the kind of reasoning that makes mathematics both rigorous and meaningful.
\\

\hfill
\noindent \textit{ \small Selected course evaluations and DRP presentation slides are available at my website \href{https://sites.google.com/view/myungsin-cho/teaching?authuser=0}{(link)}}
\end{document}