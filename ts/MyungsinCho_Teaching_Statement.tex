\documentclass[11pt]{article}
\usepackage[letterpaper,margin=1in]{geometry}
\usepackage{hyperref}

\usepackage{newtxtext, kotex}

\usepackage{fancyhdr}
%\fancyhf{} % sets both header and footer to nothing
\renewcommand{\headrulewidth}{0pt}


\usepackage[textwidth=1in,textsize=small,colorinlistoftodos]{todonotes} % todo 노트

\pagestyle{fancy}
\lhead{Myungsin Cho\\ }
%\chead{\LARGE Teaching Statement}
%\rhead{\univ \\ Ph.D. Mathematics}

% section 폰트 바꿔주는거
\usepackage{titlesec}
\titleformat{\section}{\normalfont\large\center}{\thesection}{.5em}{}
%\titleformat{\subsection}[runin]{\normalfont\bfseries}{\thesubsection}{.5em}{}

\newcommand{\yourname}{Myungsin Cho}

\newcommand{\univ}{Indiana University}

\newcommand\quelle[1]{{%
      \unskip\nobreak\hfil\penalty50
      \hskip2em\hbox{}\nobreak\hfil #1%
      \parfillskip=0pt \finalhyphendemerits=0 \par}}
      
% active learning 으로 improve할 수 있는 점들이 있음

\begin{document}
\begin{center}
\Large{Teaching Statement}
\end{center}

During my time in graduate school, I have taught a variety of lower-division courses, gaining experience with students from diverse backgrounds and learning styles. I have also led independent study projects through the Directed Reading Program (DRP), which further shaped my approach to individualized instruction. Through these experiences, I have come to appreciate the importance of creating a supportive learning environment where students feel encouraged to engage with challenging material. Ultimately, my goal is to help students develop logical reasoning and deductive thinking skills through mathematics.

\paragraph{Adapting to Different Learning Environments and Fostering Engagement}\quad \\
Teaching at two universities with vastly different academic environments—Indiana University and Seoul National University—has given me valuable insight into adapting my approach for diverse student populations while consistently fostering active engagement. As a graduate instructor, my primary responsibility was leading problem-solving sessions for Calculus courses. Regardless of the setting, I aimed to create an active learning environment. I consistently posed questions during sessions to gauge understanding, encouraged students to voice their approaches and questions, and created opportunities for peer discussion around challenging problems. My goal was always to make the classroom a space where students felt empowered to participate and tackle difficulties collaboratively.

The specific focus of this engagement, however, was tailored to the student body. At Indiana University, a large public university with students across a wide range of academic experiences, my adaptation involved emphasizing fundamental concepts and building confidence. I focused on selecting problems that directly reinforced key ideas from lectures and designed homework and quizzes to create a coherent learning narrative. The active learning techniques helped ensure students were grasping the basics and felt comfortable seeking clarification. Building on my years of experience as a TA at IU, I eventually took on the role of primary instructor of Calculus 1, and the semester was a great success. The withdrawal rate for my section was significantly lower than the course average, which I take as a sign that students felt supported and engaged in their learning through this adapted, active approach.

In contrast, at Seoul National University, a prestigious school in Korea with exceptionally talented students highly proficient in problem-solving, my focus shifted. While still employing active techniques like questioning and discussion, the emphasis was less on the mechanics and more on guiding students to articulate their deeper thought processes. The goal was to have them explain not just how they arrived at a solution, but why they chose a particular approach and how different techniques connected logically, thereby refining their advanced reasoning skills through active discourse.

\paragraph{Teaching Math as a Coherent Story} \quad \\
My experiences adapting to different student populations solidified my belief in presenting mathematics not just as a set of tools, but as a coherent story centered on conceptual understanding. The idea I emphasized most when I taught Calculus 1—the one I consistently reinforced and put the greatest effort into conveying—was that even in calculus, there is a unifying narrative. The central theme I built the course around was approximation. Rather than presenting functions, limits, derivatives, and integrals as disconnected topics, I showed how they all fit into this broader framework: functions model the world, differentiation simplifies functions locally, integration accumulates quantities, and Taylor series approximate functions with polynomials.

In my lectures, I emphasized how mathematical concepts are interconnected, showing students that understanding definitions and visual representations is the most effective way to minimize memorization. I wanted them to see that when they grasp the fundamental ideas, many results follow naturally. For instance, few students truly understand the definition of a circle beyond recognizing its shape. However, once they internalize that a circle is the set of all points equidistant from a center, deriving its equation becomes almost effortless. Similarly, defining trigonometric functions using the unit circle immediately leads to fundamental identities without the need for rote memorization.

The power of this conceptual approach became especially clear with students who had prior, yet incomplete, exposure to the material. For instance, many of my Calculus 1 students had taken AP Calculus in high school but hadn't passed the placement test. Throughout the semester, several shared that while the topics were familiar, they were finally grasping the underlying connections and the 'why' behind the structure, rather than just memorizing isolated techniques. These moments of clarity, seeing students connect the dots for the first time, powerfully reinforced my conviction that a strong conceptual foundation is the key to genuine mathematical learning.

The impact of focusing on the ``why'' behind the mathematics, rather than just the ``how,'' also became clear during my time substituting lectures at IU. On occasion, instructors entrusted me with substituting for the main lecture—a valuable opportunity that required adapting my teaching from guiding problem sessions to delivering core content with an emphasis on intuition and connections. For instance, when I substituted for a Calculus 3 class, I taught Motion in Space: Velocity and Acceleration, focusing on building intuition and connecting the abstract mathematics to physical reality. Later, a physics student shared that this particular lecture, with its emphasis on conceptual understanding, played a crucial role in their decision to pursue a double major in mathematics.

% \paragraph{Flexible and Fair Assessment Strategies}\quad \\
% I believe in offering students multiple opportunities to demonstrate their learning. Assessments should not be one-time events, and I strive to provide various chances for students to learn from their mistakes and improve. For instance, I allow students to revisit and correct errors made on exams, awarding partial credit for corrections, as I believe that growth is often most evident in how we respond to our missteps. This approach encourages continuous improvement and reduces the anxiety that often accompanies high-stakes testing.

% In addition to traditional assessments, I also incorporate extra-credit assignments tailored to meet individual student needs. For students who are eager to explore beyond the syllabus, I provide class project to engage with deeper mathematical topics. For those who struggle with the basics, I assign additional problems to help them strengthen their foundation. This flexible assessment approach allows me to recognize each student’s unique journey through mathematics, ensuring that they feel supported regardless of their starting point.

\paragraph{Mentoring and Individualized Instruction}\quad \\
Beyond the classroom, I have extensive experience mentoring students one-on-one. I have guided independent study projects through the Directed Reading Program (DRP), covering topics that are often difficult to incorporate into standard curricula and provided supplemental tutoring for students taking upper-level courses.

At Indiana University, I mentored two students through three separate DRP projects. The first student had completed an advanced topology course and wanted to explore more specialized topics. Over two semesters, we studied topological K-theory and simplicial homotopy theory. Beyond conveying mathematical knowledge, I focused on fostering a positive and supportive mentoring relationship. This was particularly important because the student faced personal difficulties related to her family circumstances. Creating a safe space where she felt comfortable discussing both academic and personal challenges was essential, and this experience strengthened my belief in mentorship that acknowledges the full humanity of each student.

The second student, after completing undergraduate algebra, wanted to explore related topics in more depth, so we studied the categorical approach to module theory. I introduced him to how the exponential law for numbers generalizes to vector spaces through the tensor product and the Hom functor. We then extended these ideas to topology, where I showed how homology theory provides concrete applications of abstract categorical concepts. Later, the student enrolled in a graduate courses on a related topic and shared that his DRP experience with me had served as a valuable bridge between undergraduate and graduate-level mathematics.

\paragraph{Conclusion} \quad \\
For most students, the mathematics course they take with me will be their final experience with the subject in their academic lives. This realization brings with it a deep sense of responsibility. Whether a student pursues advanced mathematics or never encounters the subject again, my hope is that their time in my class leaves them with a sense of accomplishment and a lasting connection to the beauty and utility of mathematics.
\\

\noindent Course evaluations and DRP presentation slides are available at my website \href{https://sites.google.com/view/myungsin-cho/teaching?authuser=0}{(link)}
\end{document}