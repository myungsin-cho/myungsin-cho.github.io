\documentclass{article}
\usepackage[letterpaper,margin=1in]{geometry}
\parindent=0in
\usepackage{hyperref}
\usepackage{multicol}

\begin{document}
\begin{center}
{\LARGE Myungsin Cho} \\[2pc]
\end{center}
\begin{minipage}[c]{0.4\linewidth}
Department of Mathematics\\ 
Indiana University Bloomington\\
831 E 3rd St, Rawles Hall \\
Bloomington, IN, 47405
\end{minipage} % no space if you would like to put them side by side
\begin{minipage}[c]{0.6\linewidth}
email:  myuncho@iu.edu \\
website: \url{https://sites.google.com/view/myungsin-cho/} \\
\\
\end{minipage}  \\[.5pc]

{\Large \bf Education} \\*[-.8pc]
\underline{\hspace{6.5in}} \\*[-.5pc]
\\
{\bf Indiana University, Bloomington}\\ 
{Ph.D. in Mathematics; Advisor: Michael Mandell} \hfill {\it  2025 July (expected)} \\
\\*[-.5pc]
{\bf Seoul National University} \\ 
{M.S. in Mathematics; Advisor: Otto van Koert} \hfill{\it 2018 August} \\*[-.5pc]
\\
{\bf Korea Aerospace University} \\
{B.S. in Engineering}\hfill{\it 2015 August}\\
\\
{\Large \bf Research Interest} \\*[-.8pc]
\underline{\hspace{6.5in}} \\*[-.5pc]
\\
My research focuses on stable homotopy theory and its interactions with other fields, particularly algebraic K-theory, number theory, equivariant algebra and combinatorics.
\\
\\
{\Large \bf Publication} \\*[-.8pc]
\underline{\hspace{6.5in}} \\*[-.5pc]
\\
{\it K-theoretic Tate-Poitou duality at prime 2}, {\bf Advances in Mathematics}, to appear. \href{https://arxiv.org/abs/2501.03460}{arXiv:2501.03460}\\
\\*[-.5pc]
{\it Algebraic extension of Tambara functors}. (in preparation)\\
\\*[-.5pc]
{\it Kobayashi hyperbolicity on analytic stacks} with G. Cho. (in preparation)\\
\\*[-.5pc]
{\it Real homological trace methods} with T. Gerhardt, L. Keenan, J. Moreno, J.D. Quigley. (in preparation)\\
\\*[-.5pc]
{\it Realizability of compatible pairs} with D. Chan, D. Mehrle, P. Sanchez Ocal, A. Osorno, B. Szczesny, and P. Verdugo. (in preparation)\\
\\ 
{\Large \bf Awards and Honors} \\*[-.8pc]
\underline{\hspace{6.5in}} \\*[-.5pc]
\\
{\bf \large Indiana University Bloomington} \\*[.5pc]
{Outstanding Thesis Award}\hfill {\it May 2025\/} \\
{College of Arts and Sciences Dissertation Research Fellowship}\hfill {\it 2024-2025\/} \\
{ Glenn Schober Award} \hfill {\it April 2023\/} \\
{ Robert E. Weber Memorial Award} \hfill {\it April 2019\/} \\
{  James P. Williams Memorial Award} \hfill {\it April 2019\/} \\
{ College of Arts and Sciences Fellowship} \hfill {\it Spring 2019\/} \\
{ Anna L. Homquest Fellowship} \hfill {\it April 2019\/} \\
\\
{\bf \large Seoul National University} \\*[.5pc]
{Lecture and Research Scholarship} \hfill {\it Fall 2016\/} \\
\\
{\bf \large Korean Mathematical Society} \\*[.5pc]
Silver Awards in 33rd University Student Contest of Mathematics \hfill {\it November 2014\/}  \\
\\
{\Large \bf Talk} \\*[-.8pc]
\underline{\hspace{6.5in}} \\*[-.5pc]
\\
{\bf  \large Invited Talk} \\*[.5pc]
{ Indiana University, Algebra seminar} \hfill {\it April 2025\/}  \\
{  AMS 2025 Spring Central Sectional Meeting \\ \hspace*{1em} Special Session on Homotopy theory and algebraic K-theory} \hfill {\it March 2025\/}  \\
{  Columbia University, Topology seminar} \hfill {\it October 2024\/}  \\
{  University of Virginia, Topology seminar} \hfill {\it October 2024\/}  \\
{ Indiana University, Topology seminar} \hfill {\it September 2024\/}  \\
{ Ohio State University, Homotopy seminar} \hfill {\it September 2024\/}  \\
{ FRG Virtual seminar} \hfill {\it January 2024\/}  \\
\\
{\bf \large Contributed Talk} \\*[.5pc]
{MathFest 2024, Contributed session: Advances in algebraic topology, Indianapolis} \hfill {\it August 2024\/}  \\
{BUGCAT Conference, Binghamton University} \hfill {\it November 2023\/}  \\
{Scissors Congruence, Algebraic K-Theory, and Trace Methods, Indiana University} \hfill {\it June 2023\/}  \\
\\
{\bf \large Student Seminar in Indiana University} \\*[.5pc]
{Equivariant stable homotopy theory} \hfill {\it Fall 2023\/} \\
{Spectra and stable homotopy theory} \hfill {\it May 2023\/} \\
{On the Quillen-Lichtenbaum conjecture} \hfill {\it April 2021\/} \\
{Lower K-theories} \hfill {\it Janunary 2021\/} \\
{Galois descent of algebraic K-theory of Witt vectors of finite length} \hfill {\it December 2020\/} \\
{On the cyclotomic trace for finite $W(k)$-algebras} \hfill {\it February 2020\/} \\
{Topological cyclic homology and cyclotomic trace} \hfill {\it October 2019\/} \\

{\bf \large Student Seminar in Seoul National University}\\*[.5pc] 
{Rational Homotopy Theory } \hfill {\it May 2019\/} \\
{Higher Category Theory} \hfill {\it January 2018\/} \\
{Homotopy and Cohomology } \hfill {\it August 2017\/} \\
{Introduction to Homotopy Theory } \hfill {\it January 2017\/} \\
{Towards Morse Homology } \hfill {\it August 2016\/} \\
{Introduction to Differential Topology } \hfill {\it Febuary 2016\/} \\
{A Brief Introduction to Simplicial Homology } \hfill {\it August 2015\/} \\
\\
{\bf \large Mini-course} \\*[.5pc]
{Homotopy theory and homological algebra, Enjoying Math (youtube channel)} \hfill {\it Winter 2022\/} \\
\\
{\Large \bf Teaching/Mentoring Experience} \\*[-.8pc]
\underline{\hspace{6.5in}} \\*[-.5pc]
\\
{\bf \large Indiana University Bloomington}  \\*[.5pc]
{\bf Teaching} \\
M311 Calculus 3  \hfill {\it Spring 2024 \/}\\
M311 Calculus 3  \hfill {\it Fall 2023 \/}\\
M211 Calculus 1 (Primary instructor)   \hfill {\it Fall 2022 \/}\\
M311 Calculus 3  \hfill {\it Spring 2022 \/}\\
M106 Mathematics of decision and beauty \hfill {\it Spring 2021}\\
M106 Mathematics of decision and beauty \hfill {\it Fall 2020}\\
M106 Mathematics of decision and beauty \hfill {\it Summer 2020}\\
M212 Calculus 2  \hfill {\it Fall 2019 \/}\\
M311 Calculus 3  \hfill {\it Fall 2019 \/}\\ \\
{\bf Mentoring Directed Reading Program} (slides available on website) \\
$\cdot$ Project title: Exponential law in vector spaces - glimpse to adjoint isomorphism theorem
 \hfill {\it Fall 2023 \/}\\
\hspace*{1em} Book: {\it An introduction to homological algebra} by J. Rotman\\
$\cdot$ Project title: Simplicial homotopy theory \hfill {\it Spring 2023 \/}\\
\hspace*{1em} Book: {\it Simplicial homotopy theory} by P. Goerss and R. Jardine\\
$\cdot$ Project title: A Ring Structure on Vector Bundles \hfill {\it Fall 2022 \/}\\
\hspace*{1em} Book: {\it Algebraic Topology from a Homotopical Viewpoint } by  M. Aguilar, S. Gitler and C. Prieto\\

{\bf \large Seoul National University} \\*[.5pc]
{\bf Teaching} \\
033.002 Calculus 2  \hfill {\it Spring 2018\/}\\
033.001 Calculus 1  \hfill {\it Summer 2017\/}\\
033.001 Calculus 1  \hfill {\it Spring 2017\/}\\
033.002 Calculus 2  \hfill {\it Fall 2016  \/}\\
033.001 Calculus 1  \hfill {\it Spring 2016\/}\\
033.002 Calculus 2  \hfill {\it Fall 2015 \/}\\ \\
{\bf Undergraduate tutoring program} \\
$\cdot$ Introduction to Analysis 1  \hfill {\it Fall 2016 \/}\\
\\
{\bf \large Korea Aerospace University} \\*[.5pc]
{\bf Undergraduate tutoring program }\\
$\cdot$ Linear algebra \hfill {\it Fall 2014 \/}\\
\\ 

{\Large \bf Service and Organizational Activities} \\*[-.8pc]
\underline{\hspace{6.5in}} \\*[-.5pc]
\\
{\bf \large Seminar Organization} \\*[.5pc]
{Topology Seminar, Indiana University} \hfill {\it 2024 - 2025\/} \\
{GSTGC 2025, Indiana University} \hfill {\it April 2025\/} \\
{Reading seminar on equivariant stable homotopy theory, Indiana University} \hfill {\it Fall 2023 \/} \\
{Reading seminar on stable homotopy theory, Indiana University} \hfill {\it May 2023 \/} \\
{Graduate student homotopy theory seminar, Indiana University} \hfill {\it Fall 2021\/} \\
{Graduate student homotopy theory seminar, Indiana University} \hfill {\it Spring 2021\/} \\
{Mathemaniac, Graduate student biannual seminar, Seoul National University}  \hfill {\it 2015 - 2018\/} \\
\\
{\Large \bf Reference} \\*[-.8pc]
\underline{\hspace{6.5in}} \\*[-.5pc]
\\
{Michael Mandell, Indiana University Bloomington, mmandell@iu.edu}  \\
{Ayelet Lindenstrauss, Indiana University Bloomington, alindens@iu.edu} \\
{Andrew Blumberg, Columbia University, andrew.blumberg@columbia.edu}  \\
{Vladimir Eiderman, Indiana University Bloomington, veiderma@iu.edu} \hfill {(teaching)} \\
{Ji-Ping Sha, Indiana University Bloomington, jsha@iu.edu} \hfill {(teaching)} \\


\end{document}